
\section{Introduction}
\label{sec:Introduction}

Migration has long been an important topic of enormous research and policy interest.
According to the most recent report on international migrant (United Nations, 2019), in the past decade, the total number of global migrants has dramatically increased from 173 million in 2000 to nearly 272 million in 2019.
Among the world’s largest immigration countries, Australia is always ranked on the list of the best immigration destinations as it is successful in building a multicultural and immigration friendly society.
Nevertheless, new migrants may still encounter some difficulties (e.g. culture and language difference) when they begin to settle in the new environment.
Therefore, concerns exist for the society, especially, for the policy makers over the migrants’ employment and occupational status.

Our study aims to explore employment and occupational assimilation patterns of Australian immigrants into local labour market. 
First, we investigate Australian immigrants’ employment assimilation by comparing the employment probability of immigrants to Australia since different years of arrival with the employment probability of Australian-born individuals. 
Second, we investigate whether there is an assimilation process in the occupational status of Australian immigrants by comparing the occupational choices of immigrants who have been in Australia for different length of time with Australian-born occupational choices. 
By doing the two sets of tests, we can examine whether the possibility of being employed and the occupational choices of Australian immigrants is different from the features of the Australian born and if there is such difference, whether the different pattern changes as the immigrants stay in Australia for a longer period of time.

We find that the predicted employment probability for an Australian immigrant is lower than that of a native-born person but the gap becomes smaller as the immigrant staying in Australia for a longer time. We also find immigrants are less (more) likely to work as managers and professionals (blue collars), relative to native-born individuals; and the predicted possibility of choosing managerial and professional jobs for immigrants increases as the number of years since their arrival increases and the predicted possibility of immigrants choosing blue-collar jobs decreases with longer period staying in Australia.

\section{Background and literature review}
\label{sec:Background and literature review}

Australia is one of the largest immigration countries with over 7.5 million migrants, which accounts for 29.7\% of the country’s population in 2019, according to the latest release on migration by Australian Bureau of Statistics (ABS, 2019).
The number of migrants living in Australia was 5.0 million in 2006, around 24.6\% of the total population in the country (ABS, 2019).
The Department of Home Affairs of Australian Government (2019) states in its report that the growing immigrant populations have been contributing to workforce growth in terms of their overall numbers entering the workforce.
This growth is also a good way to offset the effect of aging on the native-born population.
However, the growing immigrants also pose some questions regarding their assimilation to the Australian labour market.

Prior literature on migration issues generally finds that new migrants normally take some time to settle into the new surroundings and assimilate to the local populations’ behaviour as time passes \citep{BIDDLE2007,Arli2018}.
\citet{BIDDLE2007} find the health of new migrants to Australia is better than the health of Australian-born populations but the difference becomes smaller when the migrants stay in Australia for a longer time.
Similarly, \citet{Arli2018} find Australian migrants have difficulties to adapt to local culture in terms of their consumption behaviour and they find among various barriers, language is a main factor that influence migrants’ adaption.
As such evidence shown in the prior literature, the employment propensity of migrants upon arrival may be lower than the employment propensity of native-born people and it may improve with additional time that the migrants live in Australia.
Also, occupational choices (e.g., professionals, white collars, blue collars) may be different depending on whether the person is born overseas or born locally and the gap may become smaller depending on the number of years that the person lives in Australia.

Prior literature on the determinants of employment propensity and occupational status generally finds individuals’ socioeconomic characteristics such as age, gender and educational level influence their occupational attainment and wages \citep{Schmidt1975,Chapman1986}.
In line with these findings, research on migrants’ occupational choice also finds these demographic characteristics are associated with migrants’ occupation status in different countries such as China, America and Spin \citep{Xing2016,Laird2015,AmuedoDorantes2007}.
\citet{AmuedoDorantes2007} find the employment propensity for a Spanish migrant is associated with the migrant’s gender, language proficiency and education level.
Accordingly, we want to test whether the effect of several individual’s socioeconomic characteristics on their employment and occupation status will also be pronounced among Australia’s migrants (or we want to control for those socioeconomic characteristics in our sample).

\section{Data}
\label{sec:Data}

We obtain data from the latest National Health Survey 2017--18 conducted by ABS.
The survey was conducted throughout Australia in a 12-month duration from July 2017 to June 2018.
Approximately 21,300 people in 16,400 private dwellings across Australia were interviewed by trained ABS interviewers randomly.
The survey collects information on personal demographic and health characteristics (e.g., age, gender, country of born, educational attainment, English proficiency and disability information) and employment status (e.g., work status, working hours, occupation), which are of our interest.
After dropping the observations with missing data (e.g., vacant information on employment and occupational status) and an age less than 15, our final sample contains 17,244 individuals and 2X.X\% are migrants to Australia.
For more details of the reliability of the 2017-18 national health surveys, see ABS (2018).

Table 1 provides the definitions for the variables used in this study.

Our dependent variables of interest are employment propensity (indicator variable sets equal to 1 if the individual is employed, and 0 otherwise) and occupation choice (indicator variable sets equal to 0 defined as “Labours”, if the individual’s occupation is categorised in “Labours” by Australian and New Zealand Standard Classification of Occupations (ANZSCO) code; 1 defined  as “Blue Collars” if the individual’s occupation is categorised in “technicians and trades workers”, “community and personal service workers” or “machinery operators and drivers” by ANZSCO; 2 defined as “White Collars” if the individual’s occupation is categorised in “clerical and administrative workers” or “sales workers” by ANZSCO; 3 defined as “Managers” if the individual’s occupation is categorised in “Managers” by ANZSCO; and 4 “Professionals” if the individual’s occupation is categorised in “Professionals” by ANZSCO.

Our independent variables are Immigrant (indicator variable sets equal to 1 if the individual is born in a country other than Australia, and equal to 0 if the individual is born in Australia) and Year since Migration (which equals the current year 2019 minus the year of arrival if the induvial was born in a country other than Australia, and equal to the individual’s age if the individual is born in Australia).

We also include some individuals’ socioeconomic characteristics as control variables such as Age, English proficiency, Sex and Qualification. All variables are defined in detail in Table 1.

Table 1 shows some key features of our sample.
We find most immigrants in our sample arrived Australia before 1986 (36.63\%) or arrived recently from 2011 - year of collection (22.37\%). 
The age, gender and disability status of the immigrants and native-born population are reasonably distributed in our sample. 
We find most immigrants and native-born populations are in the age group between 20 - 75 years old in our sample. 
Around half of the people in our sample are females. 
More than 65\% of the immigrants and native-born people in our sample have no disability or long-term health condition. 
However, in terms of the educational level and English proficiency, we find some difference between the immigrant group and the native group. 
Specifically, more than 60\% of the immigrants in our sample reported they have Year 12 or equivalent education level, while only around 36\% of the native-born people reported they have Year 12 or equivalent education level.
Over 95\% of the native-born people reported they have the highest English-speaking level, while only 66\% of the immigrants reported they have the highest English-speaking level.
Overall, on average, the educational level of the immigrants in our sample is higher than that of the native people, however, the English level of the immigrants is lower than that of the native people.

Table 1 also shows the employment rates of immigrants and natives with different socioeconomic characteristics. 
In general, we find the employment rates of immigrants in different socioeconomic groups are lower than that of the native people in the same socioeconomic groups. 
We find higher employment rate (around 80\%) among immigrants aged between 30 - 50 years old relative to immigrants in other age groups. 
However, the employment rate among native-born people whose age falls between 30 - 50 is around 85\%, which is also higher compared to that of immigrants. 
We also find immigrants who speaker better English have a higher employment propensity relative to immigrants who have lower English-speaking level.
Overall, the employment rate in our sample is 53.7\% for native-born people and 46.1\% for immigrants. 