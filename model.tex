
\section{Model}
\label{sec:model}

First, we would like to test the relationship between immigrants‘ assimilation and employment status.
Assimilation is measured by the number of years since the immigrant's first arrival in Australia ($\mathit{Year\ since\ Migration}$).
Employment status is captured by a dummy variable which set to one if the respondent is employed and zero otherwise ($\mathit{Employment}$).
Probit model with robust standard errors is selected to control for heteroskedasticity.
More importantly, we select several control variables to assure that immigrants' assimilation is not associated with omitted variables and personal characteristics such as education level ($\mathit{Education}$), age ($\mathit{Age}$), English level ($\mathit{English}$), gender ($\mathit{Sex}$), qualification ($\mathit{Qualification}$), and health status ($\mathit{Health}$).
In addition, we add both quadratic form of $\mathit{Year\ since\ Migration}$ and quadratic form of $\mathit{Age}$) in our model as we predict that the relationships between employment and assimilation or age are U-shaped.
In sum, the Probit regression is expressed in equation \eqref{eq:1}.

\begin{equation}
  \label{eq:1}
  \begin{aligned}
  \mathrm{Pr}[\mathit{Employment}_i=1|\pmb{X}_i]
  = \Phi \Bigl[
    &\beta_0
    + \beta_1 \mathit{Immigrant}_i
    + \beta_2 \mathit{Year\ since\ Migration}_i \\
    & + \beta_3 \mathit{Year\ since\ Migration}_i^2
    + \beta_4 \mathit{Education}_i
    + \beta_5 \mathit{Age}_i \\
    &+ \beta_6 \mathit{Age}_i^2
    + \beta_7 \mathit{English}_i
    + \beta_8 \mathit{Sex}_i
    + \beta_9 \mathit{Qualification}_i \\
    &+ \beta_{10} \mathit{Health}_i
    + \mu_i
    \Bigr]
  \end{aligned}
\end{equation}

Then, we examine whether immigrants' assimilation affects their occupation choice.
In order to do this, we classify jobs into five categories which are labor workers (the base case), blue collar workers ($\mathit{Blue}$), white collar workers ($\mathit{White}$), managers ($\mathit{Manager}$), and professionals ($\mathit{Prof}$).
Participant $i$ chooses job $j$ when the utility of this choice ($\pmb{U}_{ij} = \pmb{X}_i \pmb{\beta}_{j} + \pmb{\varepsilon}_{ij}$) is the greatest.
As all variables are case-specific, multinomial logit (MNL) model is preferred.
In this model, we assume that only differences in utility ($\pmb{\varepsilon}_{ij} - \pmb{\varepsilon}_{ik}$) matter and overall scale of utility is irrelevant.
Thus, the probability of participant $i$ selecting job $j$ from $J$ choices is expressed in equation \eqref{eq:2} and the log odds ratio between choice $i$ and the base choice is illustrated in \eqref{eq:3}.

\begin{equation}
  \label{eq:2}
  \mathrm{P}_{ij}
  = \mathrm{Pr}(y_i = j)
  = \frac{
    \exp (\pmb{X}_i \pmb{\beta}_j)
  }{
    1 + \sum_{l=1}^{J-1}(\pmb{X}_i \pmb{\beta}_l)
  }
\end{equation}

\begin{equation}
  \label{eq:3}
  \log \biggl(
  \frac{\mathrm{P}_{ij}
  }{
    \mathrm{P}_{i0}
  }  \biggr)
  = \pmb{X}_i \pmb{\beta}_j
\end{equation}

Considering the fact that different jobs have different properties which is likely to affect participants' choice, we add two alternative-specific variables $\mathit{Minimum\ Wage}_j$ (which is the state minimum hourly wage rate for job $j$) and $\mathrm{Average\ Wage}_j$ (which is the average state hourly wage rate for job $j$).
In this situation, conditional logit (CL) model is better because it allows for both case-specific and alternative-specific variables.
In addition, we clustered the standard errors by state because separate state has different levels of wage and job opportunities.

One assumption of the MNL or CL model is that choices are independent of irrelevant alternatives (IIA).
But this assumption may not hold in our research given that different jobs have separate requirements such as minimum education level and minimum working experiences.
We use the Hausman test to check whether the IIA assumption is violated or not.
We consider mixed logit (MXL) model to allow for the alternative specific constants (ASCs) to be normally distributed random parameters by state when the IIA assumption does not hold.

%%% Local Variables:
%%% mode: latex
%%% TeX-master: "ECON7205_Grp_Assign"
%%% End:
